% \usepackage{graphicx}
\usepackage{fancyhdr} % activamos el paquete
\pagestyle{fancy} % seleccionamos un estilo
\geometry{headheight=1in, margin=1in, top=1.5in}

% http://www.estudiaradistancia.com.ar/logos/original/logo-universidad-nacional-de-lujan.png
\lhead{\includegraphics[width=2cm]{images/logo-unlu.png}} % texto izquierda de la cabecera
% \chead{TEXTO} % texto centro de la cabecera
\rhead{\textbf{
  UNLu - Departamento de Ciencias Básicas\\
  Teleinformática y Redes\\
  Curso \the\year
}} % texto derecha de la cabecera
% \lfoot{TEXTO} % texto izquierda del pie
% \cfoot{\includegraphics[width=11cm]{heado2}} % imagen centro del pie
\cfoot{\thepage} % numero de pagina al centro
% \rfoot{TEXTO} % texto derecha del pie
\renewcommand{\headrulewidth}{0.3pt} % grosor de la línea de la cabecera
\renewcommand{\footrulewidth}{0.3pt} % grosor de la línea del pie

%\definecolor{InlineCodeFGColor}{HTML}{990000} % color de letra para codigo inline
%\definecolor{InlineCodeBGColor}{HTML}{F4F4F4} % fondo del texto codigo inline

%\let\oldtexttt\texttt

% establece color de letra Y FONDO para codigo en linea
%\renewcommand{\texttt}[1]{
%  \colorbox{InlineCodeBGColor}{\textcolor{InlineCodeFGColor}{\oldtexttt{#1}}}
%}

% establece solo color de letra para codigo en linea
%\renewcommand{\texttt}[1]{
%  \textcolor{InlineCodeFGColor}{\oldtexttt{#1}}
%}

%\let\oldverbatim\verbatim

%\renewcommand{\verbatim}{
%  \color{InlineCodeFGColor}
%  \oldverbatim
%}
